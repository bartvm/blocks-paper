\documentclass[twoside,11pt]{article}

% Any additional packages needed should be included after jmlr2e.
% Note that jmlr2e.sty includes epsfig, amssymb, natbib and graphicx,
% and defines many common macros, such as 'proof' and 'example'.
%
% It also sets the bibliographystyle to plainnat; for more information on
% natbib citation styles, see the natbib documentation, a copy of which
% is archived at http://www.jmlr.org/format/natbib.pdf

\usepackage{jmlr2e}
\usepackage{hyperref}

% Heading arguments are {volume}{year}{pages}{submitted}{published}{author-full-names}

\jmlrheading{1}{2015}{?-??}{5/15}{??/??}{Bart van Merri\"{e}nboer, Dzmitry
Bahdanau, Jan Chorowski, Vincent Dumoulin, Dmitriy Serdyuk, David
Warde-Farley, and Yoshua Bengio}

% Short headings should be running head and authors last names

\ShortHeadings{Blocks and Fuel}{Van Merri\"{e}nboer et al.}
\firstpageno{1}

\begin{document}

\title{Blocks and Fuel: GPU-accelerated neural networks for large data}

\author{\name Bart van Merri\"{e}nboer \email bart.van.vanmerrienboer@umontreal.ca \\
        \addr Montreal Institute for Learning Algorithms, University of Montreal, Montreal, Canada
        \AND
        \name Dzmitry Bahdanau \email d.bahdanau@jacobs-university.de \\
        \addr Jacobs University, Bremen, Germany
        \AND
        \name Jan Chorowski \email jan.chorowski@ii.uni.wroc.pl \\
        \addr University of Wroc\l aw, Wroc\l aw, Poland
        \AND
        \name Vincent Dumoulin \email dumouliv@iro.umontreal.ca \\
        \name Dmitriy Serdyuk \email serdyuk.dmitriy@gmail.com \\
        \name David Warde-Farley \email wardefar@iro.umontreal.ca \\
        \name Yoshua Bengio \email yoshua.bengio@umontreal.ca \\
        \addr Montreal Institute for Learning Algorithms, University of Montreal, Montreal, Canada}

\editor{?}

\maketitle

\begin{abstract}%   <- trailing '%' for backward compatibility of .sty file
  We introduce two Python frameworks to train neural networks on large
  datasets: \emph{Blocks} and \emph{Fuel}. \emph{Blocks} is based on the
  Theano, a linear algebra compiler with
  CUDA-support~\citep{Bastien-Theano-2012,bergstra+al:2010-scipy}. It
  facilitates the training of complex neural network models by providing
  parametrized Theano operations, attaching metadata to Theano's symbolic
  computation graph, and providing an extensive set of utilities to assist
  training the networks, e.g.\ training algorithms, logging, monitoring,
  visualization, and serialization. \emph{Fuel} provides a standard format for
  machine learning datasets. It allows the user to easily iterate over large
  datasets, performing many types of pre-processing on the fly.
\end{abstract}

\begin{keywords}
  Neural networks, GPGPU, large-scale machine learning
\end{keywords}

\section{Introduction}

\emph{Blocks} is a framework that builds on Theano to facilitate the training of
neural networks. It is being developed by the Montreal Institute of Learning
Algorithms (MILA) at the University of Montreal. Its focus lies on quick
prototyping of complex neural network models. \emph{Fuel} 

\section{History}

% Pylearn2 has this section, do we need it?
% Maybe a comparison with other frameworks here?

\begin{itemize}
  \item Pylearn2: More general (machine learning vs.\ neural networks), abstracts
    away Theano
  \item Lasagne, Keras: Stronger focus on easy implementation of existing models
    than the development of new models
  \item scikit-learn: Less plug-and-play, more research-oriented
\end{itemize}

\section{Blocks}

Theano is a popular choice for the implementation of neural networks (see
e.g.~\cite{Goodfellow-et-al-ICML2013, Pascanu-et-al-ICML2013}). Blocks and many
other libraries, such as Pylearn2~\cite{pylearn2_arxiv_2013}, build on Theano
by providing reusable components that are common in neural networks, such as
linear transformations followed by non-linear activations, or more complicated
components such as LSTM units. In Blocks these components are referred ot as
\emph{bricks} or ``parametrized Theano operations''.

Bricks take the form of a set of parameters in the form of Theano shared
variables, for example the weight matrix of a linear transformation or the
filters of a convolutional layer. Bricks use these parameters to transform
symbolic Theano variables, potentially in multiple ways. % Example: Softmax cost vs. output

Bricks can contain other bricks within them. This effectively introduces a of
hierarchical structure on top of the flat computation graph defined by Theano,
which makes it easier to address and configure complex models programmatically.

The parameters of bricks can be initialized using a variety of schemes that are
popular in the neural network literature, such as sparse initialization,
orthogonal initialization for recurrent weights, etc.

Large neural networks can often result in Theano computation graphs containing
hundreds of variables and operations. Blocks does not attempt to abstract away
this complex graph, but to make it manageable by annotating variables in the
graph. Each input, output, and parameter of a brick is annotated as such.
Variables can also be annotated with the role they play in model, such as
\emph{weights}, \emph{biases}, \emph{filters}, etc. A series of convenience
tools were written that allow users to filter the symbolic computation graph
based on these annotations, resulting in an approach that allows for powerful
queries such as ``Apply weight noise to all weights that belong to an LSTM
unit whose parent is a brick with the name \ldots''

\section{Fuel}

Fuel's goal is to provide a common interface to a variety of data formats and
published datasets such as MNIST, CIFAR-10, ImageNet, etc.\ while making it easy
for users to write an interface to new datasets. Fuel allows for efficient ways
of iterating over these datasets (sequentially, shuffled, minibatches, etc.).
It also provides a variety of on-the-fly preprocessing methods such as random
cropping of images, creating n-grams from text files, and the ability to
implement many other methods easily.

Blocks relies on Fuel for its data interface, but Fuel can easily be used by
other machine learning frameworks that interface with datasets.

\section{Serialization and checkpointing}

The training of large, deep neural networks can often take days or even weeks.
Hence, regular checkpointing of training progress is important. Blocks aims to
make the resumption of experiments entirely transparent, even across platforms,
while ensuring the reproducibility of these experiments.

This goal is complicated by shortcomings in Python's \textsc{Pickle}
serialization module, which is unable to serialize many iterators, which Fuel
heavily depends on in order to iterate over large datasets efficiently. To
circumvent this we reimplemented the \textsc{itertools} from the Python
standard library to be serializable.

As a result, Blocks experiments are able to be interrupted in the middle of a
pass over the dataset, serialized, and resumed later, without affecting the
final training results.

\section{Documentation and community}

Blocks and Fuel are well documented, with both API documentation and tutorials
available online. Two active mailing
lists\footnote{\url{https://groups.google.com/d/forum/blocks-users} and
\url{https://groups.google.com/d/forum/fuel-users}} support users of the
libraries. A seperate
repository\footnote{\url{https://github.com/mila-udem/blocks-examples}} is
maintained for users to contribute non-trivial examples of the use of Blocks.
Implementations of neural machine translation models
(NMT,~\cite{bahdanau2015neural}) and the Deep Recurrent Attentive Writer
(DRAW,~\cite{gregor2015draw}) model are publicly available examples of
state-of-the-art models succesfully implemented using Blocks.

\section{NOTES}

Some topics to discuss?

\begin{itemize}
  \item Serialization/resumability
  \item ``Bricks'' i.e.\ parametrized Theano operations
  \item Graph management through annotation, hierarchy
  \item Implementations of GRU, LSTM, attention mechanism, convolutional
  networks
  \item Monitoring non-Theano variables, aggregation, plotting results
\end{itemize}

\begin{itemize}
  \item Automatic downloading and converting of files
  \item Standard HDF5 format that conveys axis semantics, splits, etc.
\end{itemize}


% Acknowledgements should go at the end, before appendices and references

\acks{The authors would like to acknowledge the support of the following
  agencies for research funding and computing support: NSERC, Calcul Qu\'{e}bec,
  Compute Canada, the Canada Research Chairs and CIFAR\@. We would also like to
  thank the developers of Theano.}


\bibliography{bibliography}

\end{document}
